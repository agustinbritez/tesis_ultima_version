
\subsection{Cifrado Asimetrico}
Los documentos digitales se firmaban digitalmente
mediante algún método de seguridad como la infraestructura de la clave pública 
la cual involucra hardware, software, políticas y procedimientos; este método también llamado  cifrado asimétrico
permite diferenciar a las entidades mediante una clave pública y una privada, la ultima es única e irrepetible 
que es usada para cifrar los mensajes que envía, para que una entidad posea su clave necesita que un tercero 
denominado autoridad de certificación encargada de emitir y revocar los certificados se los entregue \cite[]{avila_implementacion_2015}.

Ejemplo de cómo funciona:
una institución envía por correo electrónico un título digital a un estudiante, con el método de cifrado asimétrico.

% ¿Cómo se puede verificar que el  título  que recibió es realmente enviado por la institución?

Lo que sucede es lo siguiente, la institución tiene en su poder una clave pública y una privada,
la clave privada solo es conocida por la institución, ella no debe revelarla. Mientras
que la clave pública es un código que cualquier individuo puede conocerla, la institución puede 
mostrar su clave pública.
Para que el título tenga validez la institución utiliza su clave privada para cifrar el título digital, 
una vez hecho esto la institución envía el documento cifrado al estudiante, donde este último
para descubrir que contiene el documento cifrado, utiliza la llave pública para descifrar 
el documento y tener acceso a  título  digital.
% El proceso explicado abstrae el proceso técnico para entender como funciona el método de cifrado.
El método válida inequívocamente a la institución porque cualquier mensaje 
cifrado con la clave privada, solamente puede ser descifrada con la clave pública, estas llaves
se complementan. Por lo tanto si el mensaje es interceptado y cambian el  contenido, al momento de cifrarlo 
no podrá ser descifrado con la clave pública de la institución, de esta manera se verifica que la institución no envió el mensaje \cite[]{garcia_rojas_implementacion_2008,avila_implementacion_2015,avanzaexportador_certificados_2009}.


Las autoridades certificadoras juegan un rol importante ya que ellas son la que emiten estas claves,
y validan la identidad de las organizaciones o personas. Por lo tanto si una entidad “A” quiere enviar un mensaje
privado a una entidad “B”, la entidad “A” debe consultar a la autoridad de certificación cual es la clave pública de 
la entidad “B” y confiar que esa es la correcta \cite[]{garcia_rojas_implementacion_2008}. 

El cifrado asimétrico no es el único método pero es bastante utilizado y uno de los más comunes \cite[]{crypto_sheinix_bitcoin_2020}.  

Utilizando este método de criptografía se puede asegurar que una entidad es quien dice ser y también que un mensaje fue
emitido por ella. Por lo tanto el método de criptografía se utiliza para validar que una entidad es la que emitió un documento digital,
pudiendo ser este algún certificado académico,  título  académico \cite[]{garcia_rojas_implementacion_2008,avila_implementacion_2015,crypto_sheinix_bitcoin_2020}. 
