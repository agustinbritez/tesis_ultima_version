\subsection{Billetera}
Una Billetera o Wallet es una herramienta de software que permite a los usuarios enviar y recibir
criptoactivos de manera relativamente sencilla, algunas también permiten interactuar con los smart contract,
se usan estas billeteras para hacer más sencillo la manera de comunicar con la Blockchain.
Un usuario con su misma clave privada puede usarla en cualquier billetera o wallet que soporte el software.
Existen diferentes tipos de billeteras \cite[]{dannen_introducing_2017,ambito_cuales_2021,rezaeighaleh_new_2019}:
\begin{enumerate}
\item Hardware wallet: Son billeteras que almacenen la clave privada y pública de manera física, similar 
a un pendrive se consideran una de las formas más seguras, porque no están conectadas directamente a internet.
\item Wallet Online: Las claves están almacenadas en un servidor.
\item Wallet Escritorio: Aplicaciones descargadas y ejecutadas desde el computador, donde se puede acceder
a los criptoactivos que se gestionan.
\item Paper Wallet:  Las claves publicas y privadas son almacenadas en un papel físico.
\end{enumerate}

Algunas de las Wallet que permiten gestionar sus criptosactivos son metamask, trezor (física), coinbase, entre otros \cite[]{dannen_introducing_2017,ambito_cuales_2021,coinsenda_tipos_2019}.