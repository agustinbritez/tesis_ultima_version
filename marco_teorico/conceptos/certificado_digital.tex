\section{Certificado Digital y Firma Digital}
Por un lado, el Certificado Digital es un documento emitido por una organización que presta un servicio (denominada autoridad de certificación), el 
documento o archivo digital contiene datos relacionado a un usuario en particular, una persona física o una organización. El certificado 
puede ser emitido en diversos formatos de archivos, la autoridad encargada reconoce y confirma los datos expuestos en el certificado 
digital, con utilización de técnicas criptográficas de clave pública y privada o de infraestructura de clave pública (Public key infrastructure) \cite[]{avila_implementacion_2015}.

La infraestructura de clave pública permite a las entidades ser autenticadas y reconocidas por otras entidades mediante los certificados 
digitales que además permiten cifrar y descifrar mensajes, firmar digitalmente y garantizar su integridad \cite[]{avila_implementacion_2015}. 

Las partes o componentes más importante de una \gls{pki} son las que se exponen a continuación \cite[]{avila_implementacion_2015}:

\begin{itemize}
    \item La \gls{ac}: encargada de generar, emitir y revocar los certificados; es la responsable que da  legitimidad a la clave pública con respecto a la  entidad. 

    \item La \gls{ra}: es la responsable de verificar si los enlace de las claves públicas corresponden a las entidades titulares.
    
    \item Los repositorios: se encargan de almacenar la información relativa a la \gls{pki}, como los repositorios que guardan las claves públicas y  otras que resguardan las listas de claves revocadas.

    \item La \gls{va}: se encarga de comprobar que los certificados digitales sean válidos.

    \item La \gls{tsa}: encargada de firmar los documentos para crear un registro específico en el momento exacto que existió.

    \item Los usuarios finales: obtienen una clave privada y una clave pública, también un certificado asociado a su clave pública. Mediante el uso de diversas aplicaciones hacen uso de la tecnología \gls{pki} para validar, cifrar y descifrar documentos.

\end{itemize}

Por otro lado, la Firma Digital es el mecanismo de criptografía de clave pública que permite dar
seguridad al receptor del documento firmado digitalmente; la firma digital se realiza con la
clave privada del remitente para cifrar la información a enviar y el receptor emplea la
clave pública del remitente para descifrarla, para así garantizar la autenticidad del origen de la informacion y verificar 
que ésta no fue modificada desde su creación. Este es un método para validar si mensajes, documentos o archivos han sido alterados.
Al método de clave pública y clave privada también se lo denomina método asimétrico \cite[]{avila_implementacion_2015}.


La aplicación del método de cifrado asimétrico puede encontrarse en el caso de que, por ejemplo,  una institución envía por correo electrónico un título digital a un estudiante. Es decir, 
la institución tiene en su poder
una clave pública y una privada,
la clave privada solo es conocida por la institución, ella no debe revelarla. Mientras
que la clave pública es un código que cualquier individuo puede conocerla, la institución puede 
mostrar su clave pública.
Para que el título tenga validez la institución utiliza su clave privada para cifrar el título digital. Una vez hecho esto, la institución envía el documento cifrado al estudiante y éste utiliza la llave pública para descifrar el documento y tener acceso al titulo digital.
El método valida inequívocamente a la institución ya que cualquier mensaje cifrado con la clave privada solamente puede ser descrifrado con la clave pública, por lo que ambas se complementan.
Por lo tanto si el mensaje es interceptado y cambian el  contenido, al momento de cifrarlo 
no podrá ser descifrado con la clave pública de la institución y de esta manera se verifica que la institución no envió el mensaje \cite[]{avila_implementacion_2015,garcia_rojas_implementacion_2008,avanzaexportador_certificados_2009}. 


Las autoridades certificadoras juegan un rol importante ya que ellas son la que emiten estas claves,
y validan la identidad de las organizaciones o personas. Por lo tanto si una entidad “A” quiere enviar un mensaje
privado a una entidad “B”, la entidad “A” debe consultar a la autoridad de certificación  es la clave pública de 
la entidad “B” y confiar que esa es la correcta \cite[]{garcia_rojas_implementacion_2008}. 

El cifrado asimétrico no es el único método pero es uno de los más comunmente utilizados para validar documentaciones digitales \cite[]{crypto_sheinix_bitcoin_2020}.  

Utilizando este método de criptografía se puede asegurar que una entidad es quién dice ser y también que un mensaje fue
emitido por ella. Por lo tanto el método de criptografía se utiliza para validar que una entidad es la que emitió un documento digital,
pudiendo ser este algún certificado académico,  título  académico \cite[]{avila_implementacion_2015,garcia_rojas_implementacion_2008,crypto_sheinix_bitcoin_2020}. 







  

