\section{Motivación}

Con la incorporación de las Tecnologías de la Información y 
la Comunicación (TIC) a la vida cotidiana cambió las 
prácticas de gestión, transacción e interacción de la información,
por lo que las actividades que se desarrollan, tanto en el 
sector privado como sector público, requieren de una constante 
incorporación de nuevas herramientas para su gestión documental \cite[]{gauchi_risso_aproximacion_2012}.

La gestión documental implica un conjunto de operaciones y 
técnicas relativas a la concepción, desarrollo, implantación y 
evaluación de los sistemas administrativos necesarios desde la 
creación de los documentos hasta su destrucción o transferencia a 
los depósitos de archivos. Para ello, a partir del desarrollo de la tecnología,
se incorporó a  los sistemas administrativos herramientas 
innovadoras para la gestión documental digital, lo cual 
requiere también métodos de validación \cite[]{drescher_Blockchain_2017,retamal_Blockchain_2017}.

A los fines de dar validez a los documentos digitales, se han 
desarrollado métodos que permiten mantener la integridad de los 
mismos y así evitar problemas tales como duplicaciones o 
alteraciones de datos   
, y así tener valor de autenticidad y certeza de que los datos no hayan sido alterados desde su rúbrica
\cite[]{drescher_Blockchain_2017,retamal_Blockchain_2017,avila_implementacion_2015}.

La tecnología  Blockchain  permite brindar beneficios tales como 
disminuir la probabilidad de modificar o realizar duplicación de 
datos sin autorización, así como también disminuir la probabilidad 
de recibir ataques informáticos \cite[]{drescher_Blockchain_2017,retamal_Blockchain_2017,badreddin_Blockchain_2018,choo_Blockchain_2020}   
; la idea detrás de la tecnología blockchain se remonta a 1991 cuando Stuart Haber y
W. Scott Stornetta describieron el primer trabajo en una cadena de bloques
criptográficamente segura. En 1992, incorporó en el diseño los árboles Merkle, lo que permitió recopilar varios documentos en un bloque.
Sin embargo, la tecnología blockchain como la conocemos hoy ganó importancia a partir de 2008, cuando se publicó el informe técnico 
“Bitcoin: un sistema de dinero en efectivo electrónico entre pares” bajo el seudónimo Satoshi Nakamoto \cite[]{alice_blockchain_2021}, y la misma puede ser 
utilizada de diferentes maneras, por lo que en la actualidad 
existen varias implementaciones de ella, manteniendo los 
beneficios de su utilización \cite[]{Blockchain_federal_argentina_bfa_2020,dannen_introducing_2017}.

En cuanto a su aplicación en la gestión documental, los datos 
almacenados en la  Blockchain, pueden ser representados por 
documentos digitales o por una identificación única que 
simbolice ese documento digital \cite[]{Blockchain_federal_argentina_bfa_2020}.

La \glsfirst{unam} genera a 
través de sus procedimientos administrativos distintos tipos 
de documentos, tales como resoluciones, ordenanzas, convenios, 
reglamentos, solicitudes, certificaciones y otros; los cuales son 
de suma importancia, por lo que la utilización de la tecnología 
 Blockchain  permitiría validar la integridad de los datos 
pertenecientes a la documentación digital que se generen en la 
institución.