\section{Problema}


Las  Secretarías de Extensión de la \gls{unam} emiten diversas documentaciones 
que avalan la realización de actividades de extensión por parte de estudiantes, 
docentes, profesionales y público en general. Sin embargo, es dable destacar que, aquellas
documentaciones digitales carecen de métodos de validación y que es necesario indagar 
acerca de tecnologías que brinden mayor seguridad  y 
asegure que  no hayan sido alteradas desde su emisión (A.L., comunicación personal, 02/10/2020).


Esta necesidad impulsa al presente desarrollo a hallar respuestas a 
los siguientes interrogantes: ¿qué tecnologías  permiten validar documentos digitales?,
¿qué antecedentes y tendencias existen para la validación de documentos digitales?,
¿cuáles son las prácticas generales que se realizan respecto a las certificaciones de actividades
de extensión en la \gls{unam}? y ¿de qué manera se podrían validar los documentos digitales emitidos 
por actividades de extensión en la \gls{unam} que permitan garantizar la inmutabilidad de los certificados?




