\chapter{Presentación del Caso} %% Analisis de como se trabaja en la UNAM Facultad de Extension.
%^%*************************************************************************
%% Presetacion 
%% Secretaría de extensión Que hacen, donde estan, que actividads realizan,.
%% Secretaría de extensión universitaria (SEU), (estatuto). 
\section{Contexto}

%Contar sobre la UNAM e ir acortando hasta llegar que la investigacion se realizara en laa Secretaria de extension de la Facultad de ciencias economicas.
% gestio documental dee la SE para cerrtificacion de documentos.
% explicar solo SE universitaria.

El estatuto describe a la \gls{unam} como una institución Universitaria de Derecho Público,
autónoma en lo académico e institucional y
autárquica en el sector económico y financiero. La institución 
tiene asiento en la provincia de Misiones de la República Argentina \cite[]{estatuto}. 

Impulsa la relación e integración con las instituciones afines,
gubernamentales y no gubernamentales de la provincia, regional,
nacional e internacional, que compartan o coincidan con sus fines 
y objetivos \cite[]{estatuto}.

Los fines de la \gls{unam} son descritos en el estatuto lo cual se
cita textualmente a continuación:

\begin{enumerate}
\item “La preservación promoción y difusión de la cultura universal con énfasis en lo nacional y
regional.”
\item “El resguardo acrecentamiento y difusión del conocimiento universal y del generado en su
propio ámbito.”
\item “La organización instrumentación y evaluación de la enseñanza-aprendizaje en los niveles de su
competencia y su articulación con los otros sectores del sistema educativo.”
\item “La aplicación del conocimiento a la solución de problemas del desarrollo humano en la provincia
, la región y el país.”
\item “El compromiso con la conservación y preservación del medio ambiente y los recursos naturales.”
\item “El de constituirse en un ámbito de formación ciudadana y ejercicio democrático.” \cite[]{estatuto}

\end{enumerate}

La \gls{unam} esta integrada por la \gls{fhycs}; \glsfirst{fce}; \gls{fceqyn}; \gls{fi} ; \gls{fayd};
\gls{fcf} 
\cite[]{estatuto}. Cada facultad de la \gls{unam} cuenta con las diferentes  Secretarías, Direcciones y Departamentos que necesiten gestionar información 
sobre los alumnos.

Prestando atención a las  secretarías las cuales varían sus nombres según la facultad, ellas son  la Secretaría Administrativa, 
Secretaría Académica, Secretaría de Investigación, Secretaría de Bienestar Estudiantil y Secretaría de Extensión.   

Todas  tienen en común  la gestión de  documentos y de información por ende deben generar nuevos documentos a partir de los 
hechos ocurridos, como ejemplo, la creación de un certificado de título académico por consecuencia de que un alumno de la facultad
finalizó el cien porciento (100\%) de su carrera. O la generación y gestión de cualquier otro documento que sea utilizado por 
la entidad como los planes de estudios, historia académica de los estudiantes, actas de exámenes, informes, entre otros (A.L., comunicación personal, 02/10/2020) \cite[]{estatuto}. 

% Se puede observar que en cada área de la universidad utilizan la gestión de información y documentación, es necesario por cuestiones 
% de optimización que la investigación límite el dominio en la cual sea aplicable el estudio por ello es recomendado realizarlo sobre un 
% área específica como la secretaría de extensión, la cual constantemente generan nuevos documentos \cite[]{larraburu_secretariextension_2020}.

La problemática que inicialmente impulsó la investigación es la validación de documentos relacionados a los estudiantes, sean certificados
o registros que puedan ser alterados. Dada la gran variedad de documentación que utiliza la universidad, se delimitó el alcance de la investigación en un área 
especifica, aún especificando un sector,
los documentos pueden ser variados, por lo tanto es necesario enfocarse en documentos relacionados a lo académico, con respecto a este tipo 
de documentación son considerados los que manejan datos de estudiantes, o participantes de eventos, excluyendo todo tipo de documentación 
relacionada al sector financiero  o asuntos administrativos (A.L., comunicación personal, 02/10/2020). %\cite[]{larraburu_secretariextension_2020}.

Las áreas seleccionadas para realizar la presente investigación son las Secretarías de Extensión de la \gls{unam} y sus respectivas facultades. 
Estas áreas tienen que gestionar documentaciónes de estudiantes y participantes de eventos por lo tanto, es un área adecuada para desarrollar el presente proyecto.

% las causas para la investigación pueden ser (relacionada a estadística) (por conveniencia explicar que solo se hace de una forma)

%% En la FCE se dictan carreras de gestion documental a la gestion. 
%  se presume (clave usar)



\section{Certificaciones de Actividades de Extensión en la \gls{fce} }
Las actividades universitarias de la \gls{unam} pretenden promover la interacción con el
medio en la cual integra, aportando al crecimiento social \cite[]{estatuto}. 

El estatuto \cite[]{estatuto} de la \gls{unam} describe que las actividades de extensión pueden implicar transferencia científico-tecnológica, 
educación permanente, difusión de actividades 
y producciones de la \gls{unam}, el desarrollo de las expresiones culturales y vinculaciones institucionales.




% \section{Secretaría de Extensión de la Facultad de Ciencias Económicas}

A un nivel jerárquico general está la  \gls{sgeu}, como lo define su nombre
es la secretaría general encargada de  gestionar y coordinar con las  secretarías de extensión de las distintas facultades \cite[]{estatuto}.

Por otra parte, la \gls{fce}  cuenta con su propia Secretaría 
de Extensión donde llevan a la práctica tareas como la generación de cursos, congresos, charlas, capacitaciones, eventos 
de cualquier características,
que permita cumplir con sus objetivos de expansión de conocimientos y crecimiento (A.L., comunicación personal, 02/10/2020).%\cite[]{larraburu_secretariextension_2020}.

El área administrativa de la Secretaría de Extensión es la encargada de realizar la gestión de las actividades que se planea efectuar.
Los eventos pueden ser iniciados por docentes, no docentes, funcionarios, secretarios o estudiantes pero el evento o actividad
debe  estar aprobada por algún instrumento.
Estos instrumentos pueden ser las disposiciones o resoluciones  que representan documentaciones que validan e impulsan 
la creación y ejecución de las actividades (A.L., comunicación personal, 02/10/2020).%\cite[]{larraburu_secretariextension_2020}.

Por otro lado, para que una actividad sea aprobada  es necesario que la 
propuesta tenga relación con los proyectos o programas planeados previamente. Esto quiere decir que las actividades deben estar justificadas y 
sus temas deben tener una relación estricta con los objetivos que la facultad propuso en sus proyectos o programas.
En el caso que se presente una propuesta de actividad que no se encuentra relacionado directamente a los proyectos o programas,
existe la posibilidad de presentarlo a las autoridades de la Secretaría de Extensión para evaluarlos y dictaminar su aprobación o rechazo (A.L., comunicación personal, 02/10/2020).% \cite[]{larraburu_secretariextension_2020}.

Una vez aprobada la actividad o el evento, el personal administrativo 
inicia con las preparaciones para relanzarlo. Se gestionan y reservan
las aulas necesarias o en caso que los participantes deban asistir 
físicamente una fecha y hora determinada, se prepara el registro
para quienes van dirigidos, ya que los eventos 
se realizan para un grupo selecto o también para todo el público interesado.
Se inicia toda la logística necesaria para el evento en particular (A.L., comunicación personal, 02/10/2020).%\cite[]{larraburu_secretariextension_2020}.

Para que los participantes de los eventos tengan un documento que constate
su presencia en las actividades,  se les entrega un certificado que dependiendo
de la situación son de asistencia al evento o exámenes aprobados (A.L., comunicación personal, 02/10/2020).%\cite[]{larraburu_secretariextension_2020}.

Depende de como se  organizó el evento y su magnitud, puede ocurrir que un evento
cuente con más de una charla y a su vez más de un curso. Por lo tanto
al planificar las actividades se opta cómo se entregarán los certificados:
uno por charla asistida o por asistencia total del evento sin detallar la actividad que realizó (A.L., comunicación personal, 02/10/2020).%\cite[]{larraburu_secretariextension_2020}. 

Los tipos de certificados académicos que se expiden en la secretaría de extensión de la \gls{fce}, son en formato papel, hojas tamaño A4 u oficio,
también certificados digitales como PDF e imágenes (A.L., comunicación personal, 02/06/2021).%\cite[]{larraburu_secretariextension_2020}. 



  
%^%*************************************************************************
\subsection{Proceso para Generar y Emitir  Certificados}

A continuación se describe el  proceso que se utiliza en la generación y emisión de los certificados 
relacionados a las actividades de la Secretaría de Extensión de la \gls{fce}; cabe aclarar que 
se explica el proceso para la certificación y no toda la gestión que se lleva a cabo
para que una actividad se realice. 

El proceso para la emisión y generación de los certificados para un evento está explicado en la entrevista de realizada al encargado de 
la Secretaría de Extensión de la \gls{fce} de la \gls{unam} (A.L., comunicación personal, 02/10/2020):%\cite[]{larraburu_secretariextension_2020} : 

\begin{enumerate}
    \item Días antes del evento se realiza el modelado de los certificados correspondiente a las actividades 
    que se planea realizar, dependiendo si los certificados serán solo por asistir al evento, se crea un modelo.
    Pero en el caso que se realice certificación por algún exámen aprobado o realizado, se crean 
    la cantidad de modelos de certificados según los exámenes a evaluar. 
  
    \item Una vez diseñados los modelos ideales para el evento se realiza el escaneo de las firmas de las autoridades de la facultad
    que representan a la institución académica, en este caso el secretario de extensión y otros responsables de ser necesarios.

    \item La firma escaneada de las autoridades se integra en los modelos de certificados, generando la documentación con la firma de las autoridades.


    \item El día del evento se registran los participantes que asisten, tomando los datos necesarios para los certificados.

    \item Al finalizar el evento se verifican los datos de las personas que se  registraron, y 
    las que asistieron a ellas. En este punto se determina quiénes de los participantes, recibirán 
    los certificados de asistencia.

    Con los datos de los participantes del evento se completan
    los certificados de asistencia de manera manual. Se agregaron los datos necesarios como 
    Apellido y Nombre, en otros casos otros datos extras como DNI, correo electrónico, 
    los datos del certificado dependerá de como se modeló el certificado.

    \item Cuando se finalizó con la carga de datos en los certificados de asistencias se entregan a los participantes.
    Depende de la planificación de la actividad, se entrega el certificado impreso en papel.
    O se envía el certificado digital por correo electrónico a cada participante.  
    
    
    
    \item Si la actividad contempla la certificación de aprobación de exámen, o 
    algún otro requerimiento específico. Se comprueba cuántos y quiénes de 
    los participantes  realizaron y aprobaron los requerimientos o el exámen.
   
    \item Una vez filtrados  los participantes que aprobaron los requerimientos o
    exámenes, se les crean manualmente (se ingresa el nombre y otros datos necesarios )
    el certificado perteneciente a cada participante que haya cumplido con las exigencias. 
    
    \item Finalizada la creación de los certificados de aprobación de exámen o 
    el requerimiento que se solicita para el certificado en particular se realiza
    la entrega del mismo modo que los certificados de asistencia. Mediante
    correo electrónico y en caso de ser físico, se contactan con los participantes para enviarles los certificados.
    
    Este último proceso de certificación se separa del certificado de asistencia.
    Porque de acuerdo al volumen de participantes y el criterio para evaluarlos, puede consumir más tiempo.
\end{enumerate} 

\subsection{Problemas Actuales}
Cuando  los participantes pretenden demostrar que asistieron o fueron parte de un evento de actualización, capacitación
o curso, ellos no tienen la manera de validar que sus documentos
digitales son los originales o si realmente fue emitido por la \gls{unam}, la única manera es con la impresión  del certificado 
y llevarlo a la Secretaría de Extensión de la \gls{fce} para  sellar  dando confirmación de que es un certificado emitido por la institución. Pero 
la validación de los mismos de manera digital no cuenta con un método definido para realizarlo (A.L., comunicación personal, 02/10/2020).%\cite[]{larraburu_secretariextension_2020}.

% En el caso que el participante desee demostrar que realizó los cursos, charlas o eventos específicos, la Secretaría de Extensión de la \gls{fce}
% debe hacerlo de manera física,  problema que no permite a los usuarios gestionar sus certificados digitales obtenidos a lo largo de todos los 
% eventos presenciados. 
% Las ventajas de realizar los certificados digitales es que el proceso de emisión y entrega es más rápido
% comparándolo al proceso físico a través de la impresión en soporte papel  de los certificados y entregándolo a cada uno de los participantes de los eventos.
% Mientras que la entrega de los certificados digitales se pueden hacer por otros medios más rápidos, como envío de correo electrónicos.

La Secretaría de Extensión de la \gls{fce} genera los certificados de manera física, lo sellan y entregan  físicamente; en los casos
que el participante de los eventos, charlas o cursos necesite demostrar que asistió a las actividades el 
proceso para validar sus certificados son lentos y con probabilidades de ser alterados. En comparación con
un proceso de validez digital que permite  ser automatizado, por lo tanto los medios de envíos son mas rápidos como por ejemplo
correos electrónicos y  brinda mayor velocidad en la validación de los certificados.
El método que utilizan para que los certificados digitales sean considerados válidos es incluir las firmas ológrafas de 
los responsables como el Secretario de Extensión o también los profesionales que son parte de la organización del evento, pero estas firmas
pueden ser copiadas y adaptadas a otros certificados de esta manera crear documentos que no fueron emitidos por la Secretaría de Extensión de la \gls{fce}.
Las entidades externas que desean averiguar si una persona realmente estuvo en el evento de interés, no tiene manera de 
comprobar la veracidad o autenticidad del certificado, por ende, el problema puede extenderse a dudar de la validez de los certificados (A.L., comunicación personal, 02/10/2020).%\cite[]{larraburu_secretariextension_2020}.
