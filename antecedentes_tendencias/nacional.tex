\section{Casos en Argentina}
\subsection{Antecedentes}


La \gls{bfa} \footnote{\href{https://bfa.ar/}{bfa.ar}} se describe a sí misma como una plataforma multiservicios, abierta y participativa, pensada para integrar servicios y aplicaciones 
sobre Blockchain.
Una iniciativa confiable y auditable que permita optimizar procesos y funcione como herramienta de empoderamiento para toda la comunidad
\cite[]{Blockchain_federal_argentina_que_nodate}.
Sobre esta plataforma se han creado diversas aplicaciones de distintos ámbitos como la publicación de documentos como el {Boletín Oficial de la República Argentina}, de empresas privadas, también aplicaciones de trazabilidad, y otras que se pueden relacionar a:
\begin{enumerate}
  
  \item El Ministerio de Educación, Cultura, Ciencia y Tecnología cuenta con el Registro Público de Graduados Universitarios que proporciona datos de egresados universitarios certificados por el Ministerio de Educación de toda la República Argentina. Gracias a la digitalización del trámite de certificación de diplomas y analíticos, y a la incorporación de  Blockchain  en el proceso, es posible autenticar la veracidad de la información contenida en el registro y que ésta sea accesible a la comunidad \cite[]{Blockchain_federal_argentina_aplicaciones_nodate}.
  
  \item  Sistema de Información Universitario (SIU). El SIU-Diaguita como el  Módulo de Compras y el de  Contrataciones y Patrimonio del SIU, son utilizados por más de 50 Universidades Nacionales y Organismos Públicos. Se incorporó el uso de  la \gls{bfa} a través  de la funcionalidad de recepción de ofertas. De esta manera, las fechas y horas de las ofertas de los proveedores  se registran en el sistema y queda certificada en la Blockchain 
\cite[]{Blockchain_federal_argentina_aplicaciones_nodate}.

\item En la Universidad Nacional de Córdoba gracias a la digitalización de los sistemas de gestión de alumnos utilizados en universidades para la carga y archivo de actas de examen, promoción y equivalencia, entre otros datos, es posible verificar dicha información por medio de  Blockchain, y garantizar al alumnado, personal administrativo y autoridades de las unidades académicas, que el sistema no puede presentar alteraciones en los registros sin que esas modificaciones sean detectadas \cite[]{Blockchain_federal_argentina_aplicaciones_nodate}.
\end{enumerate}


\subsection{Tendencias}
Por otro lado el gobierno Argentino impulsa un proyecto de ley relacionada a las criptomonedas y activos digitales,
el objetivo es crear un marco regulatorio integral, aplicable a las transacciones y operaciones civiles y 
comerciales de criptoactivos  y permitir
el crecimiento del ecosistema local  \cite[]{dagostino_exclusivo_nodate}.

El artículo describe que la promulgación de esta ley permitirá:
\begin{enumerate}
    \item El Estado pueda determinar qué \glsplural{proyecto_cripto} autoriza y cuáles no en base criterios legales que hoy no existen.

    \item Las empresas tengan modelos de negocios que cumplan con ciertos criterios, como una  Blockchain pública o casos de uso.

    \item Se puedan representar tenencias de acciones, una propiedad, etcétera.

\end{enumerate}

La definición de un criptoactivo posibilitará que las capitales de riesgos internacionales 
que pretenden invertir tengan  seguridad jurídica y la certeza destinado los fondos \cite[]{dagostino_exclusivo_nodate}. 

