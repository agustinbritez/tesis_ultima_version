\section{Antecedentes}
%%------------------- Enfocarme solo en antecedentes 
%%de la tecnología y la certificacion (no hay informacion)-------------------

%%cuento un poco como se haciean los primeros certificados con el metodo de criptografia asimetrica.




% \subsection{Antecedentes en Blockchain}
La  Blockchain surge con Bitcoin en el año 2008, este es presentado
como una solución al doble gasto y un medio de pago puramente electrónico \cite[]A{nakamoto_bitcoin_2008},
luego surgen copias de esta red, pero cuando Ethereum hizo presencia  con
el enfoque diferente a Bitcoin, expande el uso de la Blockchain
con la incorporación de los smart contract \cite[]{ethereum_que_2020}.  %Estos son programas alojados en la Blockchain,
%que pueden ser ejecutados por cualquier nodo de la red a cambio de un pago \cite[]{ethereum_que_2020}.
No tardaron en surgir otras  Blockchain similares a la de Ethereum, como EOS, 
NEO, entre otros
cada una tiene su particularidad en sus objetivos, la manera de utilizarlos y protocolos.

% La ventaja de usar la  Blockchain es que todo los datos almacenados en ella son totalmente
% accesibles, visibles, y permiten trazabilidad por lo tanto es  auditable \cite[]{drescher_Blockchain_2017}.
% Absolutamente todo dato almacenado 
% es válido, porque el protocolo asocia una dirección o address, con algún dato especifico, el 
% mismo puede ser manipulado (pero los valores anteriores seguirán existiendo en el registro) como lo permita el smart contract, el poseedor de algún dato
% puede manejarlo según su voluntad cambiando estados del programas pero sin poder eliminar todo el historial de cambios realizados.

Desde el surgimiento de Ethereum, se  desarrollaron aplicaciones que permiten
representar activos del mundo real, siendo estos títulos de automotores, terrenos o propiedades,
títulos académicos, u otros instrumentos que representen valores para las personas.
Desde el despliegue de Ethereum 1.0 en el año 2015 se crearon  aplicaciones de video juegos,
financieras, redes sociales, arte digital, entre otros. El mundo del  Blockchain es un mar de proyectos e innovaciones,
por ende se hará foco en el área de educación académica, especialmente
en la validación de certificados, títulos o documentaciones relacionadas a esta
área (\cite[]{drescher_Blockchain_2017,cheng_Blockchain_2018}).

Al explorar las \gls{dapp} de las distintas  Blockchain la mayoría de ellas
están basadas en un ecosistema que permita a los usuarios obtener una recompensa o beneficio 
monetario digital, en esta investigación no se pretende enfocarse en el sector de las criptomonedas,
sino utilizar el poder de la  Blockchain para otros beneficios, por ende se describen los antecedentes
de esta  Blockchain aplicados a sectores académico.

% \subsubsection{BlockCerts}\label{sssec:blockcerts}




% \subsubsection{OpenCerts}\label{sssec:opencerts}
% OpenCerts  es una aplicación que permite a las entidades como escuelas, universidades, gestionen los certificados de sus alumnos
% de manera segura y sin intermediarios utilizado la tecnología  Blockchain.
% La aplicación no almacena los datos privados de los alumnos, utilizan las claves públicas referenciada a los usuarios. La función de las organizaciones o entidades es gestionar los certificados de los alumnos 
% asociándolos a su clave pública. Los estudiantes sólo podrán consultar, descargar y compartir 
% sus certificados. El administrador válida que las entidades dadas de alta sean correctas.
% Esta aplicación esta desplegada en una de las redes de prueba de Ethereum denominada la red Rinkeby \cite[]{opencerts_gestion_nodate}.

\subsubsection{Casos internacionales}

La tecnología es muy utilizada en la administración pública en países como Estonia donde
tiene un modelo de gobierno electrónico que es la identidad digital, con ella los estonios tienen sus datos
que lo identifican eléctricamente, permitiéndo acceder a servicios del País y viajar por la Union Europea.   
Tienen un sistema de ciudadanía e-Residenc para los extranjeros. De esta manera permitiendo a los
extranjeros vivir en otros países. Usan sus documentos de identidad electrónico para editar y revisar 
documentos fiscales, solicitar beneficios de seguridad social y obtener servicios bancarios (\cite[]{brys_cadena_2019}).

Estonia no es el único país que se beneficia de la tecnología  Blockchain. Ucrania usa un sistema 
eAuction 3.0 usado para el alquiler o ventas de bienes del Estado para combatir 
la corrupción y disminuir la burocracia.
En Suecia hay un proyecto que permite almacenar transacciones inmobiliarias
de forma que todas las contrapartes – los bancos, los agentes, los compradores y vendedores  pueden 
tener la oportunidad de seguir el proceso de la implementación del acuerdo después de su finalización.
Estos no son los únicos países que lo ponen en practica también  Georgia, Grecia y Honduras aplican la tecnología
Blockchain \cite[]{brys_cadena_2019}.

\subsubsection{\glsfirst{bfa}}
La \gls{bfa} se describe a sí misma como una plataforma multiservicios, abierta y participativa, pensada para integrar servicios y aplicaciones sobre Blockchain.
Una iniciativa confiable y auditable que permita optimizar procesos y funcione como herramienta de empoderamiento para toda la comunidad (\cite[]{Blockchain_federal_argentina_que_nodate}).
Sobre esta plataforma se han creado diversas aplicaciones de distintos ámbitos como la publicación de documentos como el {Boletín Oficial de la República Argentina},de empresas privadas, también aplicaciones de trazabilidad, y otras que se pueden relacionar a :

\begin{enumerate}
  
  \item El Ministerio de Educación, Cultura, Ciencia y Tecnología cuenta con el Registro Público de Graduados Universitarios que proporciona datos de egresados universitarios certificados por el Ministerio de Educación de toda la República Argentina. Gracias a la digitalización del trámite de certificación de diplomas y analíticos, y a la incorporación de  Blockchain  en el proceso, es posible autenticar la veracidad de la información contenida en el registro y que ésta sea accesible a la comunidad (\cite[]{Blockchain_federal_argentina_aplicaciones_nodate}).
  
  \item  Sistema de Información Universitario (SIU) El SIU-Diaguita como el  Módulo de Compras y el de  Contrataciones y Patrimonio del SIU, son utilizado por más de 50 Universidades Nacionales y Organismos Públicos. Se incorporó el uso de  la \gls{bfa} a través  de la funcionalidad de recepción de ofertas de esta manera, las fechas y horas de las ofertas de los proveedores  se registran en el sistema y queda certificada en la BLockchain 
(\cite[]{Blockchain_federal_argentina_aplicaciones_nodate}).

\item En la Universidad Nacional de Córdoba gracias a la digitalización de los sistemas de gestión de alumnos utilizados en universidades para la carga y archivo de actas de examen, promoción y equivalencia, entre otros datos, es posible verificar dicha información por medio de  Blockchain, y garantizar al alumnado, personal administrativo y autoridades de las unidades académicas, que el sistema no puede presentar alteraciones en los registros sin que esas modificaciones sean detectadas (\cite[]{Blockchain_federal_argentina_aplicaciones_nodate}).


\end{enumerate}



  % separar mi prepuestas con los antecedentes?

  % elementos destacados, como que tecnología usan S-SHA 256 o 
  %la cantidad de nodos que aseguran la seguridad


 %las universidades dependen de su entidad superior (MInisterio de educación). 
