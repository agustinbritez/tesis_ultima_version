\chapter{Conclusiones}

Los documentos digitales, sin distinguir el tipo información que contiene, requieren
de mecanismos o sistemas que permitan validarlos íntegramente. En el desarrollo del 
presente trabajo se expuso que existen diversas prácticas para respaldar el contenido
de documentos digitales  tales como los certificados digitales y firma digital pero carecen
de los resguardos que sí posee la tecnología Blockchain.

Tanto  a nivel internacional, nacional y provincial existen antecedentes y una fuerte tendencias
de utilizar la tecnología Blockchain para el resguardo y validación de documentaciones digitales 
y diversos activos, lo que puede aplicarse en los documentos digitales 
que respalden la realización de actividades de extensión 
en la \gls{unam}.

A efectos de buscar una solución al problema planteado, se desarrolló un sistema 
que permite a la entidad sellar en la Blockchain los documentos digitales en su poder
para que  posteriormente usuarios interesados consulten sobre la validez 
de los mismos, todo lo cual refleja la inmutabilidad de los datos y/o si existen
modificaciones en los documentos digitales (conforme los ensayos realizados). 


Por todo lo expuesto existe evidencia suficiente para no rechazar la hipótesis planteada
en el presente trabajo sin perjuicio de que existen futuras líneas de investigaciones,
tales como: 

\begin{enumerate}
    \item Investigación de  Blockchain para uso académico con bajos costos (con el fin de reconocer 
    las  Blockchains que permitan desplegar sistemas académicos o soluciones que permitan aprovechar 
    la tecnología  Blockchain).
    \item Análisis de documentos digitales originales con métodos de validaciones, a fin de  conocer cúantos documentos digitales 
    utilizan,  métodos que respalde su integridad, permitiendo incentivar el uso de sistemas como  Blockchain para fortalecer la seguridad de los documentos.
    \item Anexar la propuesta del sistema de validación con sistema de gestión documental, para generar y validar  los documentos digitales de manera
    automatizadas en la Secretaría de Extensión de la Facultad de Ciencias Económicas de la \gls{unam}, (esto permitirá reducir el proceso que realiza la Secretaría
    para generar los documentos y anexando a la propuesta de sistema de validación, tendrían los documentos respaldados por la  Blockchain).   
\end{enumerate}




