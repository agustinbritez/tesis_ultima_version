\documentclass[
final,
titlepage,
twoside,
openright,
onecolumn,
spanish,
a4paper,
12pt]{book}



% Configuracion de los margenes de la página
% Ver: https://es.sharelatex.com/learn/Page_size_and_margins
\usepackage{geometry}
 \geometry{
 a4paper,
 left=4cm, right=2.5cm,
 top=2.5cm, bottom=2.5cm
  }

% Tipografia 
\usepackage{palatino}    %	Palatino 
% \usepackage{lmodern} 	   %	Latin Modern Roman % \usepackage{tgtermes}    %	TeX Gyre Termes 
% \usepackage{tgpagella}   %	TeX Gyre Pagella 
% \usepackage{tgbonum}	   %	TeX Gyre Bonum 
% \usepackage{tgschola}    %	TeX Gyre Schola 
% \usepackage{mathptmx}    %	Times 
% \usepackage{utopia}      %	Utopia 
% \usepackage{fourier}     %	Fourier 
% \usepackage{bookman}     %	Bookman 
% \usepackage{charter}     %	Charter 
% \usepackage{lmodern}     %	Latin Modern Sans Serif 
% \usepackage{tgadventor}  %	TeX Gyre Adventor 
% \usepackage{tgheros}     %	TeX Gyre Heros 
% \usepackage{helvet}      %	Helvetica 
% \usepackage{lmodern}     %	Latin Modern Sans Typewriter 
% \usepackage{tgcursor}    %	TeX Gyre Cursor 
% \usepackage{courier}     %	Courier 
\usepackage{cclicenses}
\usepackage{ccicons} % Iconos Creative Commons
\usepackage{pifont} % Símbolos especiales ZapfDingbats: https://upload.wikimedia.org/wikipedia/commons/5/56/LaTeX-dingbats.png
\usepackage{type1cm} % Use scalable, PostScript Type 1 versions of the Computer Modern fonts.
\usepackage{amsfonts}

% Interlineado
\usepackage{setspace}
\onehalfspacing  %\doublespacing
\renewcommand{\baselinestretch}{1.5} 

% Separacion entre párrafos
%\setlength\parskip{\medskipamount}
\setlength\parskip{\bigskipamount}

% Sangría en la primera línea
\setlength\parindent{50pt}

\usepackage{pdfpages} % Para importar archivos pdf (la portada)

% Tamaño de la fuente del pie de imagen
\usepackage{caption}
\captionsetup{font=footnotesize}

%\usepackage{apacite}
\usepackage[spanish, es-tabla, activeacute]{babel}
%\usepackage{ucs} 
\usepackage[utf8]{inputenc}
\usepackage[T1]{fontenc}
\usepackage[anythingbreaks]{breakurl} % Para cortar las URL largas

\usepackage{graphicx,psfrag,fancyhdr,layout,appendix,subfigure}
%\{setspace}         
%\usepackage{multicol}         

% Fancy LaTeX chapter styles
% Options: Sonny, Lenny, Glenn, Conny, Rejne, Bjarne, Bjornstrup
\usepackage[Sonny]{fncychap}
% \usepackage[Bjornstrup]{fncychap}
% \usepackage[Sonny]{fncychap}
%\usepackage[Bjornstrup]{fncychap}

\ChNumVar{\fontsize{76}{80}\usefont{OT1}{pzc}{m}{n}\selectfont}
\ChTitleVar{\raggedleft\huge\sffamily\bfseries}  
\ChNameVar{\bfseries\Large\sf}
%\ChNumVar{\Huge}
%\ChTitleVar{\bfseries\Large\rm}
%    \ChRuleWidth{1pt}
%\ChNameUpperCase
%\ChTitleUpperCase

\usepackage{csquotes}

% para las tablas
%\usepackage[table,xcdraw]{xcolor}
%\usepackage{lscape}
%\usepackage{booktabs}

% Para comentar segmentos del texto
\usepackage{comment}

\usepackage{float}
\usepackage{textcomp}
\usepackage{etoc}
\usepackage{titlesec}
\usepackage{makeidx} 
\usepackage[nottoc]{tocbibind}
\widowpenalty=100000
\clubpenalty=100000
\raggedbottom
\hyphenpenalty=9000
\tolerance=100

% Para agregar puntos en los niveles de la tala de contenidos
\usepackage{tocloft}
% \renewcommand{\cftpartleader}{\cftdotfill{\cftdotsep}} % Partes
% \renewcommand{\cftchapleader}{\cftdotfill{\cftdotsep}} % Capítulos
% \renewcommand{\cftsecleader}{\cftdotfill{\cftdotsep}} % Secciones


% Leave the smallest margins on A4 paper.
%\usepackage{a4wide}


% Package for formatting sequence alignments
% Most people won't need this.
%\usepackage{texshade}

% Para que las referencias sean hipervínculos
\usepackage{hyperref}
\hypersetup{
colorlinks=true,
linkcolor=black,
filecolor=magenta,
urlcolor=cyan,
citecolor=black,
backref=page,
pdfpagemode=FullScreen,
}
\urlstyle{same}
\usepackage[anythingbreaks]{breakurl} % Para cortar las URL largas

% Para mostrar los números de secciones las referencias cruzadas
\usepackage{cleveref}
\crefname{enumi}{position}{positions}

\makeindex


% Bibliografia 
%%%%%%%%%%%%%%%%%%%%%%%%%%%%%%%%%%%%%%%%%%%%%%
\usepackage[
%backend=bibtex, 
backend=biber,
style=authoryear, % apa
citestyle=authoryear, % citestyle=apa,
%bibstyle=apa,
sortcites=true,
sorting=nyt,
uniquename=init,
% giveninits=true, % Abreviar los nombres
autopunct=true,
autolang=hyphen,
hyperref=true,
abbreviate=false,
backref=true
]{biblatex} 

\addbibresource{Bibliografia.bib}
\DeclareLanguageMapping{spanish}{spanish-apa}
\defbibheading{bibempty}{}

% Para APA 6  
% \documentclass[a4paper,man,natbib]{apa6}  
% \usepackage[style=apa,sortcites=true,sorting=nyt,backend=biber]{biblatex}


% \input{Silabeo}	% Agregar en este archivo las palabras que sistema no sabe separar en silabas

% Glosario
\usepackage[xindy]{glossaries} 
\makeglossaries
\usepackage[xindy]{imakeidx}
\makeindex
%Las definiciones se muestran solo si en el cuerpo del documento uso 
% \gls{clave} ejemplo \gls{LaTex}

\newglossaryentry{olografa}
{
name=ológrafa/o,
description={La RAE lo define como: Escrito de mano del autor, autógrafo. }
}
\newglossaryentry{google_form}
{
name=Formulario de Google  (Google Form),
description={Formularios de Google te permite planificar eventos, 
enviar 
una encuesta, hacer preguntas a tus estudiantes o recopilar otros 
tipos de información de forma fácil y eficiente. Se trata de crear un documento para la recogida de 
datos, ya sea de forma personalizada o anónima.}
}
\newglossaryentry{criptomoneda}
{
name=Criptomoneda,
description={Una criptomoneda, criptodivisa o criptoactivo es un medio
digital de intercambio que utiliza criptografía fuerte para asegurar las
transacciones financieras, controlar la creación de unidades adicionales
y verificar la transferencia de activos. Las criptomonedas son un tipo
de divisa alternativa y de moneda digital. Las criptomonedas tienen un
control descentralizado, en contraposición a las monedas centralizadas y
a los bancos centrales \cite[]{brys_cadena_2019}},
plural={Criptomonedas}
}
\newglossaryentry{token}
{
name=Token,
description={Un token en la  Blockchain  representa un activo digital, sigue una logica programable que 
genera un sentido al token},
plural={Tokens}
}

\newglossaryentry{proyecto_cripto}
{
name=Proyecto Cripto,
description={Son los proyectos desarrollados con tecnología  Blockchain y generalmente especifican un proyecto que utiliza criptomonedas },
plural={Proyectos Criptos}
}
\newglossaryentry{TimeStamp}
{
name=Marca de tiempo (TimeStamp),
description={La marca de tiempo es un valor numérico que representa un tiempo determinado},
plural={TimeStamps}
}

\newglossaryentry{faucet}
{
name=Grifo o Faucet,
description={Son medios por lo cuales se envían criptomonedas de manera gratuita (\cite[]{dannen_introducing_2017})},
plural={Faucets}
}
\newglossaryentry{abi}
{
name=ABI,
description={ Application Binary Interface (Interfaz Binaria de Aplicación) es el modo estándar de interactuar con los Smart Contract del 
ecosistema Ethereum, desde a fuera y adentro de la Blockchain \cite[]{ethereum_especificacion_nodate}.   },
plural={ABIs}
}








% % Acrónimos % % %
%\newacronym{<label>}{<abbrv>}{<full>}

%% --------------------Facultad ------------------------------------------
\newacronym{unam}{UNaM}{Universidad Nacional de Misiones}
\newacronym{fceqyn}{FCEQyN}{Facultad de Ciencias Exactas, Químicas y Naturales}
\newacronym{fce}{FCE}{Facultad de Ciencias Económicas}
\newacronym{fhycs}{FHyCS}{Facultad de Humanidades y Ciencias Sociales}
\newacronym{fayd}{FAyD}{Facultad de Arte y Diseño}
\newacronym{fcf}{FCF}{Facultad de Ciencias Forestales}
\newacronym{fi}{FI}{Facultad de Ingeniería}
\newacronym{sgeu}{SGEU}{Secretaría General de Extensión Universitaria }

%% ------------------ Blockchain  --------------------------------
\newacronym{bfa}{BFA}{ Blockchain  Federal Argentina}
\newacronym{dapp}{Dapp}{Aplicación Descentralizada (Decentralized application)}
\newacronym{pow}{PoW}{Prueba de Trabajo o Proof of Work}
\newacronym{pos}{PoS}{Prueba de Participación o Proof of Stake }
\newacronym{poa}{PoA}{Prueba de Autoridad o Proof of Authority }


%% ---------------- certificado digital ------------------------
\newacronym{lvm}{LVM}{Logical Volume Manager}
\newacronym{pki}{PKI}{Infraestructura de Clave Pública o Public Key Infrastructure}
\newacronym{ac}{AC}{Autoridad de Certificación o Certificate Authority}
\newacronym{ra}{RA}{Autoridad de Registración o Registration Authority}
\newacronym{va}{VA}{Autoridad de Validación o Validation Authority}
\newacronym{tsa}{TSA}{Autoridad de Sellado de Tiempo o TimeStamp Authority}


